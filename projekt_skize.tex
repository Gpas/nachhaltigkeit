\documentclass[10pt,paper=a4,final]{scrartcl}
\usepackage[utf8]{inputenc}
\usepackage{geometry}           %allows us to specify the 'seitenrand'
\usepackage{graphicx}           %package used to include graphics
\usepackage{caption}
\usepackage{subcaption}
\usepackage{hyperref}           %used to make klickable links
\usepackage{listings}
\usepackage{tabularx}
\usepackage{pdflscape}
\usepackage[figuresright]{rotating}
\usepackage{nameref}
\usepackage{longtable}
\usepackage{enumitem}
\usepackage{ngerman} % Make the document German
% Make the document German
\usepackage{ngerman}
\usepackage{fancyhdr}
\usepackage{lipsum}
\usepackage{mdwlist}
\usepackage[table,usenames,dvipsnames]{xcolor}
\usepackage{currfile}


\hypersetup{
    colorlinks,
    citecolor=black,
    filecolor=black,
    linkcolor=black,
    urlcolor=black
}

\title{Nachhaltigkeit in den Ingenieurswissenschaften}
\subtitle{Projekt Skize}
\author{Stefan Andonie \& Pascal Grütter}
\date{\today{}}

% Make title and author accessible in the header/footer
\makeatletter
  \let\Title\@title
  \let\Subtitle\@subtitle
  \let\Author\@author
\makeatother

\pagestyle{fancy}

%\geometry{a4paper, top=20mm, right=20mm, bottom=20mm, left=20mm}

\fancyhf{}      %delete default values

\setlength{\headheight}{1.5cm}
\setlength{\headwidth}{\textwidth}      %header and footer width equal the text width

% Header and Footer offset
\newlength\FHoffset
\setlength\FHoffset{1.5cm}
\addtolength\headwidth{2\FHoffset}
\fancyheadoffset{\FHoffset}
\fancyfootoffset{\FHoffset}

%\lhead{\Author}
\lhead{\itshape \Author \newline \Subtitle \newline \Title \newline BZG2309a}
%\rhead{\Title}
\rhead{ \includegraphics[width=4cm]{bfh_logo} }
\fancyfoot[LE,LO]{\textbf{BFH - Berne University of Applied Sciences} \newline
\currfilename }
%\fancyfoot[RE,RO]{\thepage}
\renewcommand{\headrulewidth}{0pt}
\renewcommand{\footrulewidth}{0.5pt}

\begin{document}

\section*{Nachhaltigkeitsanalyse einer Wärmepumpe}

\subsection*{Zusammenfassung}

\subsection*{Kontext \& Umfeld}

\subsection*{Zielsetzungen}

\subsection*{Methodik}

\subsection*{Erwartete Resultate}

  Es ist zu erwarten, dass die Wärmepumpe im direkten Vergleich mit einer
  Ölheizung besser abschneidet im Bezug auf die Aspekte der Nachhaltigkeit.
  
  Wir erwarten, dass eine Wärmepumpe Energieeffizienter arbeitet und auch
  umweltfreundlicher sein wird, da keine Fossilen Brennstoffe zu Einsatz kommen.

\subsection*{Erwartete Schwierigkeiten / Herausforderungen}

\subsection*{Projektteam}

  \begin{description}
    \item[Andonie Stefan] Zuständig für ?
    \item[Grütter Pascal] Zuständig für ?
  \end{description}

\subsection*{Aspekte der Nachhaltigkeit}

  \begin{description}
    \item[Ökologisch] Reduktion von Treibhausgasemissionen im Transport und
    der Energiegewinnung.
    \item[Ökonomisch] Langfristige Einsparung der Kosten durch Reduzierung des
    externen Energiebedarfs wie Strom und fossile Brennstoffe.
    \item[Sozial] Autonome und Dezentrale Energiegewinnung.
  \end{description}

\subsection*{Zeitplan}

\subsection*{Quellen}

\subsection*{Offene Fragen}


\end{document}
