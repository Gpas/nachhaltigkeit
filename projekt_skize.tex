\documentclass[10pt,paper=a4,final]{scrartcl}
\usepackage[utf8]{inputenc}
\usepackage{geometry}           %allows us to specify the 'seitenrand'
\usepackage{graphicx}           %package used to include graphics
\usepackage{caption}
\usepackage{subcaption}
\usepackage{hyperref}           %used to make klickable links
\usepackage{listings}
\usepackage{tabularx}
\usepackage{pdflscape}
\usepackage[figuresright]{rotating}
\usepackage{nameref}
\usepackage{longtable}
\usepackage{enumitem}
\usepackage{ngerman} % Make the document German
% Make the document German
\usepackage{ngerman}
\usepackage{fancyhdr}
\usepackage{lipsum}
\usepackage{mdwlist}
\usepackage[table,usenames,dvipsnames]{xcolor}
\usepackage{currfile}


\hypersetup{
    colorlinks,
    citecolor=black,
    filecolor=black,
    linkcolor=black,
    urlcolor=black
}

\title{Nachhaltigkeit in den Ingenieurswissenschaften}
\subtitle{Projekt Skizze}
\author{Stefan Andonie \& Pascal Grüter}
\date{\today{}}

% Make title and author accessible in the header/footer
\makeatletter
  \let\Title\@title
  \let\Subtitle\@subtitle
  \let\Author\@author
\makeatother

\pagestyle{fancy}

%\geometry{a4paper, top=20mm, right=20mm, bottom=20mm, left=20mm}

\fancyhf{}      %delete default values

\setlength{\headheight}{1.5cm}
\setlength{\headwidth}{\textwidth}      %header and footer width equal the text width

% Header and Footer offset
\newlength\FHoffset
\setlength\FHoffset{1.5cm}
\addtolength\headwidth{2\FHoffset}
\fancyheadoffset{\FHoffset}
\fancyfootoffset{\FHoffset}

%\lhead{\Author}
\lhead{\itshape \Author \newline \Subtitle \newline \Title \newline BZG2309a}
%\rhead{\Title}
\rhead{ \includegraphics[width=4cm]{bfh_logo} }
\fancyfoot[LE,LO]{\textbf{BFH - Berne University of Applied Sciences} \newline
\currfilename }
%\fancyfoot[RE,RO]{\thepage}
\renewcommand{\headrulewidth}{0pt}
\renewcommand{\footrulewidth}{0.5pt}

\begin{document}

\section*{Nachhaltigkeitsanalyse einer Wärmepumpe}

\subsection*{Zusammenfassung}

Wir zeigen die grobe Funktionsweise einer Sole-Wasser Wärmepumpe auf. Dabei wird auf die verschiedenen Aspekte der Nachhaltigkeit eingegangen. Mit dem Vergleich mit einer Ölheizung wollen wir zeigen, dass Wärmepumpen effizienter und nachhaltiger sind.

\subsection*{Kontext \& Umfeld}

Ich (Pascal Grüter) habe selber eine Wärmepumpe bei mir zuhause. Interesse wieviel sparsamer und nachhaltiger diese Heizung nun gegenüber einer Ölheizung wirklich ist.

\subsection*{Zielsetzungen}

Erklärung des Aufbaus einer Wärmepumpe.
Vergleich mit Ölbrennkessel.

\subsection*{Methodik}

Funktionsweise der Wärmepumpe und Ölbrennkessel beschreiben. Aspekte der Nachhaltigkeit aufzeigen. \\
Für den Vergleich gehen wir von einem definierten Raum aus (z.B. 100 m$^{3}$) und berechnen, wie viel Energie die beiden Heizungen brauchen, um diesen Raum um ein Grad Celsius zu erwärmen.

\subsection*{Erwartete Resultate}

  Es ist zu erwarten, dass die Wärmepumpe im direkten Vergleich mit einer
  Ölheizung besser abschneidet im Bezug auf die Aspekte der Nachhaltigkeit.
  
  Wir erwarten, dass eine Wärmepumpe energieeffizienter arbeitet und auch
  umweltfreundlicher sein wird, da keine fossilen Brennstoffe zum Einsatz kommen.

\subsection*{Erwartete Schwierigkeiten / Herausforderungen}

Da wir beide Informatik studieren, werden wir gewisse Funktionen der Heizung evt. vereinfacht darstellen müssen. Sonst verstehen wir es selber nicht.

\subsection*{Projektteam}

  \begin{description}
    \item[Andonie Stefan (andos1)]
    \item[Grüter Pascal (grutp1)]
  \end{description}

\subsection*{Aspekte der Nachhaltigkeit}

  \begin{description}
    \item[Ökologisch] Reduktion von Treibhausgasemissionen im Transport und
    der Energiegewinnung.
    \item[Ökonomisch] Langfristige Einsparung der Kosten durch Reduzierung des
    externen Energiebedarfs wie Strom und fossile Brennstoffe.
    \item[Sozial] Autonome und Dezentrale Energiegewinnung.
  \end{description}

\subsection*{Zeitplan}

25.04: Präsentation
Bis Ende April: Funktionsweise und Aspekte der Wärmepumpe
Bis Mitte Mai: Vergleich mit Ölheizung

\subsection*{Quellen}

http://www.waltermeier.com/de/walter-meier-in-der-schweiz/ \\
https://de.wikipedia.org/wiki/Heizöl \\
https://de.wikipedia.org/wiki/Wärmepumpe \\

\subsection*{Offene Fragen}

Physkalische Definition der Heizleistung?

\end{document}
