\chapter{Nachhaltigkeits Rosette}
\label{chap:rosette}

Für beide Systeme wird nun eine Nachhaltigkeitsrosette erstellt. Dabei besteht die Rosette aus drei Kategorien mit je vier Kriterien. Die Kategorien setzen sich zusammen aus Umwelt, Wirtschaft und Gesellschaft. Unsere Einteilung hat folgende Kriterien ergeben:

\section{Kriterien}

\begin{itemize}
  \item \textbf{Umwelt}
    \begin{itemize}
      \item Treibhausgas Emissionen
      \item Verbrauch von fossilen Ressourcen
      \item Wirkungsgrad und Effizienz
      \item Nachhaltigkeit verbrauchter Ressourcen
    \end{itemize}
 \item \textbf{Wirtschaft}
    \begin{itemize}
      \item Preis
      \item Montagekosten
      \item Betriebskosten
      \item Verfügbarkeit
    \end{itemize}
 \item \textbf{Gesellschaft}
    \begin{itemize}
      \item Akzeptanz
      \item Lärmbelastung
      \item Trend
      \item ?
    \end{itemize}
\end{itemize}

\section{Bewertung}
\subsection{Überblick}

\begin{center}
\begin{tabular}[c]{|p{0.25 \textwidth}|p{0.25 \textwidth}|p{0.25 \textwidth}|}

  \hline
  \textbf{\Large{Kriterium}} &
  \textbf{\Large{Wärmepumpe}} &
  \textbf{\Large{Ölbrennwertkessel}} \\ \hline

  \multicolumn{3}{|c|}{\textbf{\Large{Umwelt}}} \\ \hline
  
  Treibhausgas Emissionen
  & 0 & 0 \\
  Energieverbrauch nicht Erneuerbarer Energie
  & 0 & 0 \\
  Wirkungsgrad und Effizienz
  & 0 & 0 \\
  Nachhaltigkeit verbrauchter Ressourcen
  & 0 & 0 \\
  \hline
  
  \multicolumn{3}{|c|}{\textbf{\Large{Wirtschaft}}} \\ \hline
  
  Preis
  & 0 & 0 \\
  Montagekosten
  & 0 & 0 \\
  Betriebskosten
  & 0 & 0 \\
  
  & 0 & 0 \\
  \hline

  \multicolumn{3}{|c|}{\textbf{\Large{Gesellschaft}}} \\ \hline

  Akzeptanz
  & 0 & 0 \\
  Lärmbelastung
  & 0 & 0 \\

  & 0 & 0 \\

  & 0 & 0 \\
  \hline

\end{tabular}
\end{center}

\subsection{Begründungen}


