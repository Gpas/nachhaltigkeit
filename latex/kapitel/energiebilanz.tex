\chapter{Energiebilanz}
\label{chap:bilanz}

Wir werden die Benötigte Energie und damit den $CO_2$ Ausstoss der Beiden
Heizungen Anhand der von uns festgelegten Heizleistung von 10 kW.
Wir nehmen an, dass eine Heizung 2300 Betriebsstunden aufweisst.

[http://www.energie.ch/heizung]
energie-experten.de

\section{Wärmepumpe}

Mit Hilfe des COP berechnen wir die Benötigte Stromleistung bei einer
Aussentemperatur von 6.6 C°.
Das Ergibt eine Stromleistung 1.95 kW.

Wir Errechnen anhand des vorher definierten Energiemixes den Gesammtverbrauch
der Energie um diesen Strom bereitzustellen und den damit einhergehenden $CO_2$
Ausstoss.
Wir erhalten 5.46 kW Gesammtverbrauch und 445.5 g $C0_2$.

Pro Jahr ergibt das 1t $CO_2$ Emissionen und einen Energieverbrauch von 12558 kWh.

\section{Ölbrennwertkessel}

Wir betrachten den Wirkungsgrad der Ölheizung.
Laut Herstellerangaben Liegt dieser bei 97\%.
Damit kommen wir auf einen Bedarf von 10.3 kW.
Nun beziehen wir noch die Produktionsenergie für das ÖL mit ein und erhalten
insgesammt 33.99 kW.
Der $CO_2$ Ausstoss liegt bei 290 g CO2 pro kwh.
[http://www.co2-emissionen-vergleichen.de/Heizungsvergleich/CO2-Vergleich-Heizung.html].
das Ergibt uns einen $CO_2$ Ausstoss von 9857 g.

Pro Jahr ergibt das 22,7 t $CO_2$ und einem Energiebedarf von 78177 kWh.


\section{Vergleich}


