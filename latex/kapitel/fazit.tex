\chapter{Diskussion}
\label{chap:fazit}

\chapter{Schlussfolgerung}

Unsere Vermutungen wurden bestätigt.
Die Solewasserwärmepumpe ist eine Nachhaltige Technologie und ist im Vergleich
mit dem Ölbrennwertkessel in fast allen Aspekten nachhaltiger.

Wir haben Herausgefunden, dass die Wärmepumpe gesamthaft weniger $CO_2$
produziert und auch weniger Energie für den Betrieb verbraucht. 
Das obwohl die Ölheiztechnik eine sehr hohe Effizienz aufweist und
als bewährte Technik gilt, die eine lange Entwicklung hinter sich hat und
obwohl Heizöl eine sehr hohe Energiedichte aufweist.
Erstaunlicher weise wird auch das produzieren von Strom (in bestimmten Ländern)
immer Nachhaltiger.

Die einzigen Nachteile die sie gegenüber dem Ölbrennwertkessel aufweist, liegt
in der Lärmemission die von dem Kompressor der Wärmepumpe verursacht wird
und der geringeren Vorlauftemperatur.
Der Ölbrennwertkessel ist flexibler einsetzbar.
Dies spielt aber eine Untergeordnete Rolle, da moderne Häuser sehr gut
isoliert werden, also eine Hohe Vorlauftemperatur gar nicht nötig ist.
