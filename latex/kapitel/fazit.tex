\chapter{Schlussfolgerung}
\label{chap:fazit}

Unsere Vermutungen wurden bestätigt.
Die Sole-Wasser Wärmepumpe ist eine nachhaltige Technologie und ist im Vergleich
mit dem Ölbrennwertkessel in vielen Aspekten nachhaltiger.

Wir haben herausgefunden, dass die Wärmepumpe gesamthaft weniger $CO_2$
produziert und auch weniger Energie für den Betrieb verbraucht. 
Obwohl die Ölheiztechnik eine sehr hohe Effizienz aufweist und
als bewährte Technik gilt, die eine lange Entwicklung hinter sich hat und
obwohl Heizöl eine sehr hohe Energiedichte aufweist.
Erstaunlicher weise wird auch das produzieren von Strom (in bestimmten Ländern)
immer nachhaltiger.

Die Nachteile die sie gegenüber dem Ölbrennwertkessel aufweist, liegt
in der Lärmemission die von dem Kompressor der Wärmepumpe verursacht wird
und der geringeren Vorlauftemperatur. Auch die hohen Montagekosten der Wärmepumpe können abschreckend wirken. Jedoch wird das durch die niedrigen Betriebskosten wieder ausgeglichen.

Durch du flexiblere Vorlauftemperatur ist der Ölbrennwertkessel flexibler einsetzbar.
Dies spielt aber eine untergeordnete Rolle, da moderne Häuser sehr gut isoliert werden und meistens auch Bodenheizschlaufen besitzen. Dadurch ist eine hohe Vorlauftemperatur gar nicht nötig. Auch wurde bei den neueren Wärmepumpen die Vorlauftemperatur verbessert und mit noch besserer Technik, sollte dieser Unterschied zur Ölheizung nur noch sehr klein sein.
