\chapter{Systemgrenze}
\label{chap:Systemgrenze}

Die Definition einer Systemgrenze dient dazu das System auf die Wesentlichen
Punkte zu reduzieren die untersucht werden sollen und die Analyse in einem
�berschaubaren Rahmen zuhalten.
Dazu m�ssen Annahmen getroffen werden und Vereinfachungen gemacht werden.
Die Aussagekraft des Resultates h�ngt stark damit zusammen wie diese getroffen
werden. Darum ist es besser ein m�glichst konkretes System anzuschauen anstatt
einen Allgemeinen Fall zu formulieren.

Unser Modell besteht aus einem System, in das Ressourcen eingef�hrt werden
(Input) und das daraus eine Nutzeinheit produziert, die aus dem System gewonnen
wird (Output).

\section{das System}

Wir betrachten in unserer Arbeit ein Heizsystem in Form einer
Solewasserw�rhmepumpe und als Vergleichssystem eine Heizung mit
�lbrennwertkessel.

\section{Input}

Wir betrachten in unserer Arbeit den laufenden Energieverbrauch der beiden
Systeme pro Jahr.
Beiden Systemen muss von aussen Energie zugef�gt werden um sie zu betreiben.
Daher k�nnen wir die Input Einheit in kWH ausdr�cken.

Die Solewasserw�rmepumpe bezieht ihre Energie haupts�chlich aus Zwei quellen,
der aus der Erde in Form von Erdw�rme und vom Stromnetz, vor allem um den
Kompressor zu betreiben und den Solewasserkreislauf in Gang zuhalten.
F�r uns ist vor allem der Stromverbrauch interessant, da dies ein Kostenpunkt
ist und damit auch Umweltbelastungen verbunden sind.
Als Grundlage nehmen wir den Schweizer Energiemix nach ...
Wir betrachten die Erdw�rme als �ffentlichesgut, das unbegrenzt und kostenlos
zur Verf�gung steht.

Ein �lbrennwertkessel bezieht seine Energie in Form von Heitz�l.
Wir k�nnen den Energieinput dadurch berechnen indem wir den Liter verbrauch
ansehen und die Energiedichte eines Liter �ls somit haben wir wider den
Energieinput in kWH pro Jahr erfasst.
F�r einen Liter Heiz�l dieses Typs wird so und so viel Energie gerechnet ...

Die Graueenergie, die bei der Herstellung und bei der Montage der Systeme
verbraucht wird nicht weiter in die Rechnung aufgenommen.
Wir nehmen an, das die Systeme im Wesentlich aus den gleichen Materialien
gefertigt sind und das  das Bohren der L�cher f�r die Erdw�rmegewinnung
nicht sehr ins Gewicht fallen wird.
Wir nehmen auch an, das das System f�r eine Lange Dauer erhalten bleibt.
Die Lebensdauer einer Heizung ist etwa die selbe wie die des Hauses, 
30 - 50 Jahre.
Da wir den Verbrauch pro Jahr berechnen w�rde sich die Graueenergie auf diesen
Zeitraum verteilen.

\section{Output}

Als Output haben wir die Nutzbare Heizleistung.
Um was handelt es sich hierbei?
bla bla ...

Eine Heizung ist ein Wasserkreislauf der mit einer bestimmten Temperatur
(R�cklauftemperatur) in das System eintritt und auf ein gewisses Temperaturniveau
gehoben wird (Vorlauftemperatur). Das erhitzte Wasser verl�sst das System und
gibt die aufgenommene Energie an die Umgebung ab.
 
Moderne Heizungsysteme werden als Fussbodenheizung realisiert.
Wir k�nnen die Ben�tigte Heizleistung f�r eine Fl�che von 100 $m^2$
berechnen.
berechnung wie Hier ...

Aus der Outputenergie k�nnen wir �ber den Wirkungsgrad R�ckschl�sse �ber
den Input anstellen.










