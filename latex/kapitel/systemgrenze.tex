\chapter{Systemgrenze}
\label{chap:Systemgrenze}

Die Definition einer Systemgrenze dient dazu das System auf die wesentlichen
Punkte zu reduzieren die untersucht werden sollen und die Analyse in einem
überschaubaren Rahmen zuhalten.
Dazu müssen Annahmen getroffen werden und Vereinfachungen gemacht werden.
Die Aussagekraft des Resultates hängt stark damit zusammen wie diese getroffen
werden. Darum ist es besser ein möglichst konkretes System anzuschauen anstatt
einen allgemeinen Fall zu formulieren.

Unser Modell besteht aus einem System, in das Ressourcen eingeführt werden
(Input) und das daraus eine Nutzeinheit produziert, die aus dem System gewonnen
wird (Output).

\section{das System}

Wir betrachten in unserer Arbeit ein Heizsystem in Form einer
Solewasserwärmepumpe und als Vergleichssystem eine Heizung mit
Ölbrennwertkessel.

\section{Input}

Wir betrachten in unserer Arbeit den laufenden Energieverbrauch der beiden
Systeme pro Jahr.
Beiden Systemen muss von aussen Energie zugefügt werden um sie zu betreiben.
Daher können wir die Input Einheit in kWH ausdrücken.

Die Solewasserwärmepumpe bezieht ihre Energie hauptsächlich aus zwei Quellen:
\begin{itemize}
\item Aus der Erde in Form von Erdwärme
\item Vom Stromnetz für die Betreibung des Kompressors
\end{itemize}

Für uns ist vor allem der Stromverbrauch interessant, da dies ein Kostenpunkt
ist und damit auch Umweltbelastungen verbunden sind.
Als Grundlage nehmen wir den Schweizer Energiemix nach ...
Wir betrachten die Erdwärme als öffentliches Gut, das unbegrenzt und kostenlos
zur Verfügung steht.

Ein Ölbrennwertkessel bezieht seine Energie in Form von Heizöl.
Wir können den Energieinput dadurch berechnen indem wir den Literverbrauch
ansehen und die Energiedichte eines Liter Öls somit haben wir den
Energieinput in kWH pro Jahr erfasst.
Für einen Liter Heizöl dieses Typs wird so und so viel Energie gerechnet ...

Die Energie, welche bei der Herstellung und bei der Montage der Systeme
verbraucht wird, werden wird nicht berücksichtigen.
Wir nehmen an, das die Systeme im wesentlichen aus den gleichen Materialien
gefertigt sind. Auch ignorieren wir die benötigte Energie für die Bohrungen der Löcher für die Erdwärmegewinnung.
Die Lebensdauer einer Heizung ist etwa die selbe wie die des Hauses, 
30 - 50 Jahre.
Da wir den Verbrauch pro Jahr berechnen würde sich die Graueenergie auf diesen
Zeitraum verteilen.

\section{Output}

Als Output haben wir die Nutzbare Heizleistung.
Um was handelt es sich hierbei?
bla bla ...

Eine Heizung ist ein Wasserkreislauf der mit einer bestimmten Temperatur
(Rücklauftemperatur) in das System eintritt und auf ein gewisses Temperaturniveau
gehoben wird (Vorlauftemperatur). Das erhitzte Wasser verlässt das System und
gibt die aufgenommene Energie an die Umgebung ab.
 
Moderne Heizungsysteme werden als Fussbodenheizung realisiert.
Wir können die Benötigte Heizleistung für eine Fläche von 100 $m^2$
berechnen.
berechnung wie Hier ...

Aus der Outputenergie können wir über den Wirkungsgrad Rückschlüsse über
den Input anstellen.










