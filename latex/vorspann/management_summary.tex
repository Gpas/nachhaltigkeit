\chapter*{Zusammenfassung}
\label{chap:zusammenfassung}

Im Rahmen des Moduls Nachhaltigkeit in den Ingenieurwissenschaften wurde eine
Nachhaltigkeitsanalyse für Heizsysteme erstellt.
Wir haben dafür zwei verschiedene Heizsysteme des Anbieters Junkers ausgewählt.
Zum einen die Solewassererdwärmepumpe STM 100-1 als aktuelle Technologie.
Und als Vergleichsystem und bisherige Technologie den Ölbrennwertkessel KUB 19-4. 

Wir bauen die Analyse schrittweise mit den Entsprechenden Methoden auf und
Daraus eine Schlussfolgerung zu Ziehen.

Mit einer SWOT-Analyse betrachten wir als erstes die qualitativen Stärken und
Schwächen der beiden Systeme. Wobei wir schon einige Nachteile der Ölheizung
identifiziert haben.

Danach können wir mithilfe der Nachhaltikeitsrosette die qualitativen Merkmale
der beiden Systeme subjektiv erfassen und direkt Miteinander vergleichen.
Wir sehen wo die Wärmepumpe gegenüber der Ölheizung noch aufholbedarf hat.
Jedoch in den Meisten Aspekten eine Bessere Leistung erbringt.

Zum Schluss Erstellen wir für beide Systeme eine Energiebilanz.
Wir betrachten den Energieverbrauch während eines Betriebsjahres und den
damit einhergehenden Ausstoss von Treibhausgas.
Die Resultate diese quantitative Methode sagen objektiv aus welches System
Energiesparender und Umweltfreundlicher ist.
Wiederum schneidet die Wärmepumpe hier besser ab.
