\chapter*{Zusammenfassung}
\label{chap:zusammenfassung}

Im Rahmen des Moduls \glqq Nachhaltigkeit in den Ingenieurswissenschaften\grqq{} haben wir zum Thema Wärmepumpe eine Nachhaltigkeitsanalyse des Energieverbrauchs erstellt. Dabei liegt der Fokus auf der Erdwärmepumpe, welche mit einem Erdwärmekreislauf Wärme aus bis zu 30-50 Meter Tiefe holt.

Dieses System wurde zum Vergleich einem Ölbrennwertkessel gegenübergestellt und in verschiedenen Kriterien bewertet. Für beide Systeme haben wir eine SWOT-Analyse, eine Nachhaltigkeitsrosette und eine Energiebilanz für den Betrieb erstellt.

Wir haben dafür zwei verschiedene Heizsysteme des Anbieters Junkers ausgewählt.
Zum einen die Sole-Wasser Erdwärmepumpe \glqq STM 100-1\grqq{} als aktuelle Technologie.
Und als Vergleichsystem und bisherige Technologie den Ölbrennwertkessel \glqq KUB 19-4\grqq{}.

Mit einer SWOT-Analyse betrachten wir als erstes die qualitativen Stärken und
Schwächen der beiden Systeme. Wobei hier schon einige Nachteile der Ölheizung
deutlich werden.

Danach können wir mithilfe der Nachhaltikeitsrosette die qualitativen Merkmale
der beiden Systeme subjektiv erfassen und direkt ,iteinander vergleichen.
Wir sehen wo die Wärmepumpe gegenüber der Ölheizung noch Aufholbedarf hat.

Zum Schluss erstellen wir für beide Systeme eine Energiebilanz.
Wir betrachten den Energieverbrauch während eines Betriebsjahres und den
damit einhergehenden Ausstoss von Treibhausgasen.

Das ganze Mini-Projekt wird abgeschlossen mit einer Schlussfolgerung und einer persönlichen Reflexion.



