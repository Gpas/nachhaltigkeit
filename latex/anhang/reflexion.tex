\chapter{Persönliche Reflexion}
\label{chap:reflexion}

\subsection{Stefan Andonie}

TODO: Nur von deiner Sicht aus beschreiben

Das Miniprojekt hat uns einen Einblick gegeben in die Methoden und Begriffe
der Nachhaltigkeit.
Im Umfang des Moduls Nachhaltigkeit in den Ingenieurwissenschaften haben wir
einiges Neues gelernt und konnten eine einige interessante Aspekte anhand
des Beispiels von Heizungssystemen genauer betrachten.
Für uns Informatiker war es spannend ein solches System zu betrachten, da wir
ansonsten eher mit Virtuellen Gütern umgehen.
Wir waren uns daher zum Teil nicht so sicher wie wir die verschiedenen Methoden
anwenden sollen, zumal die Heizungsbranche ihr eigene Fachsprache hat.
Bei der Energiebilanz haben wir mehrfach das Vorgehen gewechselt.
Zum Schluss hat es uns vor allem geholfen das System einzugrenzen und zu
vereinfachen.
Rückblickend können wir festhalten, dass es mithilfe der verschiedenen Methoden
gelungen ist, eine einigermassen Zutreffende Aussage zu machen.

\subsection{Pascal Grüter}

Ich habe bei mir zuhause eine Erdwärmepumpe und es interessierte mich, ob diese Art Heizungssystem überhaupt nachhaltig ist. Durch diese Arbeit habe ich mehr erfahren über die Deklaration von Heizungen, ihrer Funktionsweise und der Nachhaltigkeit von Ölheizungen und Wärmepumpen.

Wie ich am Anfang vermutet habe, ist die Wärmepumpe um einiges nachhaltiger als eine Ölheizung. Das ist natürlich vor allem wegen dem verwendeten Öl. Die Effizienz ist auch bei Ölheizungen sehr gut, was mich erstaunt hat.

Durch diese Arbeit habe ich auch einen Einblick in die Aspekte der Nachhaltigkeit bekommen und gemerkt, dass eine vollständige Nachhaltigkeitsbewertung eines Produktes sehr komplex und aufwändig werden kann.
In Zukunft werde ich noch mehr auf die Nachhaltigkeit von Produkten und Prozessen achten als vorher und auch kritischer damit umgehen.